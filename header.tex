%
%	Headerdatei der Bachelorarbeit
%

%	KOMA-Script Dokumentklasse "scrbook"
\documentclass[12pt, pdftex, a4paper, english, bibliography=totoc, listof=totoc, parskip=half]{scrreprt}

%%%% === Verzeichnisse (TOC, LOF, LOT, BIB) ===
   %liststotoc,      % Tabellen & Abbildungsverzeichnis ins TOC
   %idxtotoc,        % Index ins TOC
   %bibtotoc,         % Bibliographie ins TOC
   %bibtotocnumbered, % Bibliographie im TOC nummeriert
   %liststotocnumbered, % Alle Verzeichnisse im TOC nummeriert 
   
  
\usepackage{color} 

%Tabellen- und Bildunterschriften formatieren
\addtokomafont{caption}{\small\bfseries}
\addtokomafont{captionlabel}{\bfseries}
 
%	Zeilenabstand
\usepackage{setspace}
\onehalfspacing

\usepackage{geometry}
\geometry{a4paper,left=35mm,right=25mm,%
bottom=25mm,top=25mm,bindingoffset=5mm,%
includehead,includefoot
}

\usepackage{varioref}
\usepackage{url}

%	Paket zum �bersetzen
\usepackage[english]{babel}

% Eingabe von Umlauten
\usepackage[latin1]{inputenc}

%	Verwenden von T1 Fonts
\usepackage[T1]{fontenc}

%	Einrichten der Quotes
\usepackage[autostyle,german=quotes]{csquotes}

\usepackage{lmodern}

%PDF Version 1.6
\pdfminorversion=6

%Einbinden von Grafiken
\usepackage{graphicx}

%Einbinden von Grafiken in Tabellen
\usepackage{picins}
\usepackage{array}

%Einbinden von Grafiken in Tabellen HIER
%\usepackage{here}

%Tabellen
\usepackage{booktabs}

\usepackage{tocbasic}

%Benutzen eines Literaturverzeichnisses
\usepackage[backend=bibtex8,style=alphabetic-verb]{biblatex}		%alphabetic %authoryear-icomp %numeric
 %numeric %authortitle-tcomp
\bibliography{lib/libarchive}

%Fu�noten durchlaufend nummerieren
\usepackage{chngcntr}
\counterwithout{footnote}{chapter}



\pagestyle{headings}
% Eigene Kopfzeile
%\usepackage{scrpage2}
%\pagestyle{scrheadings}
%\automark[section]{chapter}

%-----------Abk�rzungsverzeichnis---------------

\usepackage[intoc]{nomencl}
% Befehl umbenennen in abk
\let\abk\nomenclature
% Deutsche �berschrift
\renewcommand{\nomname}{Abk�rzungsverzeichnis}
% Punkte zw. Abk�rzung und Erkl�rung
\setlength{\nomlabelwidth}{.20\hsize}
\renewcommand{\nomlabel}[1]{#1 \dotfill}
% Zeilenabst�nde verkleinern
\setlength{\nomitemsep}{-\parsep}
\makenomenclature

%-----------------------------------------------

%----------------------------------
%----------Neue Kommandos----------
%----------------------------------

\newcommand{\myabb}[1]{Abbildung~\ref{#1}}
\newcommand{\mytab}[1]{Tabelle~\vref{#1}}

%----------------------------------
%----------------------------------
%----------------------------------


\usepackage{enumitem}

%------------M�-einbinden----------

\usepackage{textcomp}





%
%	EOF
%